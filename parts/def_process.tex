% !TEX root = ../main.tex
Точкові випадкові процеси є основним поняттям, що досліджується в роботі.
Наведемо початкові означення з \cite{Resnick_1987}.
В межах цього пункту, якщо не сказано інакше,
$E$ --- підмножина скінченновимірного евклідового простору,
$\mathcal{E} = \mathcal{B}(E)$ --- борелева $\sigma$-алгебра підмножин $E$.
Для точкової міри $\mu$ позначимо 
$S_\mu = \left\{x \in E : \mu\left(\{x\}\right) \neq 0\right\}$ ---
множину атомів.
\begin{definition}[\cite{Resnick_1987}, ст. 124]
    Точкова міра $\mu$ називається \emph{простою},
    якщо $\mu\left(\{x\}\right) \leq 1$ для всіх $x \in E$.
\end{definition}
Позначимо $M_p(E)$ множину усіх точкових мір, визначених на $E$,
а $\mathcal{M}_p(E)$ --- найменшу $\sigma$-алгебру
підмножин $M_p(E)$, що містить усі множини виду
$\left\{
    \mu \in M_p(E) : \mu(F) \in B
\right\}$ для всіх $F \in \mathcal{E}$ і $B \in \mathcal{B}\left([0, +\infty]\right)$.
Також зафіксуємо деякий ймовірнісний простір --- трійку
$\left(\Omega, \mathcal{A}, \mathbb{P}\right)$, де
$\Omega$ --- простір елементарних подій, $\mathcal{A}$ ---
$\sigma$-алгебра підмножин $\Omega$, а $\mathbb{P}$ --- міра на цьому просторі,
що додатково задовольняє умову $\mathbb{P}\left(\Omega\right) = 1$.
\begin{definition} 
    \emph{Точковий випадковий процес} $N$ --- вимірне відображення
    з ймовірнісного простору $\left(\Omega, \mathcal{A}, \mathbb{P}\right)$
    в $\left(M_p(E), \mathcal{M}_p(E)\right)$.
\end{definition}
Якщо зафіксувати $\omega \in \Omega$, то $N(\omega, \cdot)$
буде точковою мірою. З іншого боку, якщо зафіксувати $F \in \mathcal{E}$,
то $N(F)$ буде випадковою величиною зі значеннями в $[0, +\infty]$.
Також, точковий процес $N$ задає ймовірнісну міру 
$P_N = \mathbb{P} \circ N^{-1} = \mathbb{P}\left[N \in \cdot \; \right]$
на $\mathcal{M}_p(E)$.

Надалі для спрощення точкові випадкові процеси
будемо називати просто \emph{точковими процесами}. Наведемо
декілька теорем, що стосуються означення точкового процесу.

\begin{theorem}[\cite{Resnick_1987}, ст. 124]
    $N$ є точковим процесом тоді і тільки тоді, коли для кожного
    $F \in \mathcal{E}$
    відображення $\omega \mapsto N(\omega, F)$
    з $\left(\Omega, \mathcal{A}\right)$
    в $\left([0, +\infty], \mathcal{B}([0, +\infty])\right)$
    є вимірним.
\end{theorem}
\begin{theorem}[\cite{Resnick_1987}, ст. 126]
    Нехай $N$ --- точковий процес на вимірному просторі
    $\left(E, \mathcal{E}\right)$, а сім'я передкомпактних множин $\mathcal{F}$
    задовольняє наступні умови:
    \begin{enumerate}
        \item $\left(A, B \in \mathcal{F}\right) \Rightarrow \left(A \cap B \in \mathcal{F}\right)$;
        \item $\mathcal{E}$ є мінімальною $\sigma$-алгеброю, що містить $\mathcal{F}$;
        \item Існує послідовність множин $E_n \in \mathcal{F}$, для якої
        $E_1 \subset E_2 \subset ...$ і $\bigcup_{n=1}^{\infty} E_n = E$.
    \end{enumerate}
    Для $k \in \mathbb{N}$ визначимо скінченновимірні розподіли
    $$
        P_{I_1,...,I_k} \left(n_1, ..., n_k\right) = 
        \P{N(I_j) = n_j, 1\leq j \leq k}
    $$
    для $I_i \in \mathcal{F}$ та цілих $n_i \geq 0$, $1 \leq i \leq k$.
    
    Тоді система скінченновимірних розподілів
    $\left\{P_{I_1,...,I_k}, k = 1,2,..., I_j \in \mathcal{F} \right\}$
    однозначно визначає розподіл $P_N$.
\end{theorem}

{\color{red}Означення незалежності процесів --- чи потрібно?}

{\color{red}Означення функціоналу Лапласа --- чи потрібно?}

Як і для випадкових величин, для
точкових процесів можна ввести поняття <<середнього значення>>.
\begin{definition}[\cite{Kallenberg_2017}, ст. 127]
    \emph{Мірою інтенсивності} або \emph{середньою мірою} точкового процесу $N$
    називається міра $\mu$, що для $F \in \mathcal{E}$ визначена як
    $$
        \mu(F) = \E N(F) = \int_{\Omega} N(\omega, F) d\mathbb{P} = 
        \int_{M_p(E)} m(F) d P_N
    $$.
\end{definition}

Наведемо приклад точкового процесу.
\begin{definition}[\cite{last_penrose_2017}, ст. 11]
    Нехай $P$ --- деяка ймовірнісна міра на $\left(E, \mathcal{E}\right)$, а
    $X_1, \dots, X_m$ --- незалежні випадкові величини з відповідним розподілом.
    Для кожного $i = 1, \dots, m$ $\delta_{X_i}$ --- це точковий процес,
    для якого $\P{\delta_{X_i}(F) = 1} = \P{X_i \in F}$,
    $\P{\delta_{X_i}(F) = 0} = \P{X_i \notin F}$ для $F \in \mathcal{E}$.
    Точковий процес $X = \delta_{X_1} + \delta_{X_2} + \dots + \delta_{X_m}$
    називається \emph{біноміальний процесом}
    з розміром вибірки $m$ та розподілом $P$. Для нього
    $$
        \P{X(F) = k} = C_m^k P(F)^k (1 - P(F))^{m-k}, \; k = 0,\dots,m, \; F \in \mathcal{E}
    $$.
\end{definition}

Перейдемо до означення процесу Пуассона, який є центральним у роботі.
\begin{definition}[\cite{Resnick_1987}, ст. 130]
    Нехай $\mu$ --- радонова міра на $\mathcal{E}$.
    Точковий процес $N$ називається \emph{процесом Пуассона} або
    \emph{випадковою мірою Пуассона} з мірою інтенсивності $\mu$, якщо $N$ 
    задовольняє наступні умови:
    \begin{enumerate}
        \item Для будь-якої $F \in \mathcal{E}$ 
        та будь-якого невід'ємного цілого числа $k$
        \begin{gather*}
            \P{N(F) = k} = \begin{cases}
                % \frac{(\mu(F))^k}{k!} \exp\left\{-\mu(F)\right\}, & \mu(F) < \infty \\
                \frac{(\mu(F))^k}{k!} e^{-\mu(F)}, & \mu(F) < \infty, \\
                0, & \mu(F) = \infty;
            \end{cases}
        \end{gather*}
        \item Для будь-якого натурального $k$, 
        якщо $F_1, \dots, F_k$ з $\mathcal{E}$ попарно не перетинаються, то
        $\left(N(F_i), 1\leq i \leq k\right)$ є незалежними в сукупності випадковими величинами.
    \end{enumerate}
\end{definition}
% {\color{red}окреме означення функція від міри? --- Resnick, ст. 127}

Як і для невипадкових точкових мір, для точкових процесів також можна ввести поняття
грубої збіжності.
\begin{definition}[\cite{Kallenberg_2017}, ст. 109]
    Нехай $\left(\xi_n, n \geq 1\right)$ --- послідовність 
    точкових процесів на вимірному просторі $\left(E, \mathcal{E}\right)$.
    Якщо $\E \varphi(\xi_n) \to \E \varphi(\xi)$ 
    для кожної обмеженої функції $\varphi$, 
    неперервної на $M_p(E)$ відносно грубої збіжності мір,
    то послідовність $\left(\xi_n, n \geq 1\right)$
    \emph{грубо збігається за розподілом}, що позначається
    $\xi_n \overset{vd}{\longrightarrow} \xi$.
\end{definition}
Наведемо критерій грубої збіжності за розподілом.
\begin{theorem}[\cite{Kallenberg_2017}, ст. 121]\label{kallenberg_th}
    Нехай $\left(\xi_n, n \geq 1\right)$ --- послідовність 
    точкових процесів на вимірному просторі $\left(E, \mathcal{E}\right)$,
    а точковий процес $\xi$ --- простий. Нехай також
    $\mathcal{U} \subset \hat{\mathcal{E}}_\xi$ --- фіксоване
    розсікаюче кільце, де $\hat{\mathcal{E}}_\xi$ позначає сім'ю
    борелевих підмножин $E$, для яких $\E \xi (\partial B) = 0$,
    а $\mathcal{I}\subset\mathcal{U}$ --- напів-кільце. 
    Тоді 
    $\xi_n \overset{vd}{\longrightarrow} \xi$ тоді і тільки тоді, коли
    \begin{enumerate}
        \item $\underset{n\to\infty}{\lim}\;\P{\xi_n(U) = 0} = \P{\xi(U) = 0}$ для $U\in\mathcal{U}$;
        \item $\underset{n\to\infty}{\limsup}\; \P{\xi_n(I) > 1} \leq \P{\xi(I) > 1}$ для $I \in \mathcal{I}$.
    \end{enumerate}
\end{theorem}
Для практичних застосувань є корисною наступна теорема про неперервне відображення.
\begin{theorem}[\cite{Resnick_2007}, ст. 42]
    Нехай $\left(\xi_n, n \geq 1\right)$ --- послідовність 
    точкових процесів на вимірному просторі $\left(E, \mathcal{E}\right)$,
    яка грубо збігається за розподілом до точкового процесу $\xi$,
    а відображення $\varphi: M_p(E) \to \R$ таке, що
    $\P{\xi \in \left\{ 
        \mu \in M_p(E) : \varphi \text{
             не є неперевною в 
        } \mu
    \right\}} = 0$.

    Тоді послідовність випадкових величин 
    $\left(\varphi(\xi_n), n \geq 1\right)$
    збігається за розподілом до $\varphi(\xi)$,
    тобто $\varphi(\xi_n) \overset{d}{\longrightarrow} \varphi(\xi)$.
\end{theorem}