% !TEX root = ../main.tex
Для чисельних перевірок доведених граничних теорем та демонстрації збіжності
необхідно користуватися якимось алгоритмом для отримання вибірок
з розподілу \eqref{ESF}. 
У \cite{Arratia} наводиться два підходи,
засновані на понятті каплінгу: побудові такого випадкового вектора
$\left(X_1, ..., X_n\right)^T$ зі значенням в
$\left\{1,...,n\right\}^n$, координати якого певним чином
будуть утворювати незалежні цикли, з яких утвориться перестановка
з потрібним розподілом (нагадаємо, в силу теореми \ref{th:perm_decomposition}
кожна перестановка однозначно представляється композицією циклів
з точністю до їх порядку). 

\subsection{Процес китайського ресторану}
Розглянемо випадкові величини $A_1, A_2, ...$ з розподілами
\begin{gather}\label{chinese_rest}
    \P{A_i = j} = \begin{cases}
        \frac{\theta}{\theta + i - 1}, & j = i, \\
        \frac{1}{\theta + i - 1}, & j = 1, 2, ..., i-1.
    \end{cases}
\end{gather}
Перший незалежний цикл починається з 1. 2 додається до цього циклу справа
(і він стає циклом $(1, 2)$) з ймовірністю $\frac{1}{\theta+1}$,
або ж починає новий цикл з ймовірністю $\frac{\theta}{\theta+1}$.
Нехай перші $k-1$ натуральних чисел вже розставлені в цикли.
Тоді $k$ або починає новий цикл з ймовірністю $\P{A_k = k} = \frac{\theta}{\theta + k - 1}$,
або додається справа від $j$ у вже наявний цикл з ймовірністю
$\P{A_k = j} = \frac{1}{\theta + k - 1}$, $j = 1,...,k-1$.
З алгоритму побудови отримуємо, що ймовірність отримати перестановку
на $\left\{1,...,n\right\}$
з $k$ циклами дорівнює
$\frac{\theta^{k-1}}{
    (\theta + 1) \dots (\theta + n - 1)
} = 
\frac{\theta^{k}}{
    \theta(\theta + 1) \dots (\theta + n - 1)
}
$, як і в формулі \eqref{ESF}.

Цикли, отримані за цим алгоритмом, впорядковані наступним чином:
перший містить 1, другий --- найменше число, яке не ввійшло в перший, і так далі.

Варто також зауважити, що цей алгоритм дозволяє отримати не просто
випадкову перестановку з розподілом $\ESF{n, \theta}$,
а послідовність перестановок з розподілами 
$\ESF{2, \theta}, ..., \ESF{n-1, \theta}$, причому два числа,
що в якийсь момент опинилися в одному циклі, завжди залишаються в ньому ж.

Розглянемо приклад для $n = 3$, який можна проілюструвати наступною діаграмою:
\begin{center}
    \begin{tikzpicture}[auto,vertex/.style={draw,ellipse,minimum width=60pt}]
        \node[vertex] (one) {$(1)$};
        \node[vertex, above right=1.5cm and 2.5cm of one] (onetwo) {$(1, 2)$};
        \node[vertex, below right=1.5cm and 2.5cm of one] (one_two) {$(1) (2)$};
        \node[vertex, right=5cm of onetwo] (onetwothree) {$(1, 2, 3)$};
        \node[vertex, below right=0.75 and 5cm of onetwo] (onetwo_three) {$(1, 2) (3)$};
        \node[vertex, above right=0.75cm and 3cm of onetwo] (onethreetwo) {$(1, 3, 2)$};
        \node[vertex, above right=0.75cm and 3cm of one_two] (onethree_two) {$(1, 3) (2)$};
        \node[vertex, right=5cm of one_two] (one_twothree) {$(1) (2, 3)$};
        \node[vertex, below right=0.75cm and 2.5cm of one_two] (one_two_three) {$(1) (2) (3)$};
        \path[-{Stealth[]}, every node/.style={sloped,anchor=south,auto=false}]
            % (one) edge node {$\frac{1}{\theta + 1}$} (onetwo)
            % (one) edge node[below] {$\frac{\theta}{\theta + 1}$} (one_two)
            % (onetwo) edge node {$\frac{1}{\theta + 2}$} (onetwothree)
            % (onetwo) edge node {$\frac{1}{\theta + 2}$} (onethreetwo)
            % (onetwo) edge node[below] {$\frac{\theta}{\theta + 2}$} (onetwo_three)
            % (one_two) edge node {$\frac{1}{\theta + 2}$} (onethree_two)
            % (one_two) edge node {$\frac{1}{\theta + 2}$} (one_twothree)
            % (one_two) edge node[below] {$\frac{\theta}{\theta + 2}$} (one_two_three);
            (one) edge node {$A_2 = 1$} (onetwo)
            (one) edge node[below] {$A_2 = 2$} (one_two)
            (onetwo) edge node {$A_3 = 2$} (onetwothree)
            (onetwo) edge node {$A_3 = 1$} (onethreetwo)
            (onetwo) edge node[below] {$A_3 = 3$} (onetwo_three)
            (one_two) edge node {$A_3 = 1$} (onethree_two)
            (one_two) edge node {$A_3 = 2$} (one_twothree)
            (one_two) edge node[below] {$A_3 = 3$} (one_two_three);
    \end{tikzpicture}
\end{center}
З цієї діаграми видно, що ймовірність отримати перестановку $(1) (2, 3)$
обчислюється як 
$\P{A_3 = 2} \cdot \P{A_2 = 1} = \frac{1}{\theta + 2} \cdot \frac{\theta}{\theta + 1}$.
Аналогічно можна перевірити, шо ймовірність отримати інші перестановки з двома циклами
теж дорівнює $\frac{1}{\theta + 2} \cdot \frac{\theta}{\theta + 1}$,
перестановку $(1, 2, 3)$ з одним циклом --- $\frac{1}{\theta + 2} \cdot \frac{1}{\theta + 1}$,
а тотожну перестановку $(1)(2)(3)$ з трьома циклами ---
$\frac{\theta}{\theta + 2} \cdot \frac{\theta}{\theta + 1}$.

\subsection{Каплінг Феллера}
Розглянемо  незалежні випадкові величини
$B_1, B_2, ...$ з розподілами
\begin{gather}
    \P{B_i = j} = \begin{cases}
        \frac{\theta}{\theta + i - 1}, & j = 1, \\
        \frac{i - 1}{\theta + i - 1}, & j = 0.
    \end{cases}
\end{gather}
Знову почнемо перший незалежний цикл з 1. Якщо $B_n = 1$,
то цей цикл закінчується, а новий починається з 2, а інакше ---
довільно (з рівними ймовірностями) обирається число одне з $n-1$ чисел,
що залишились, і додається до цього циклу справа від 1.
На наступному кроці, якщо $B_{n-1} = 1$, то поточний цикл
закінчується, а новий починається з найменшого
натурального числа, що ще не потрапило до циклів, а інакше ---
довільно обирається одне з $n-2$ чисел і додається до поточного циклу справа.
Цей процес повторюється, доки не утвориться перестановка,
що буде реалізацією $\ESF{n, \theta}$.

Якщо порівнювати цей каплінг з процесом китайського ресторану,
то можна помітити, що $B_i = \mathds{1}\left\{A_i = 1\right\}$, де
$A_i$ визначено \eqref{chinese_rest}. Різницею є те, що
каплінг Феллера використовує $A_1,...,A_n$ в зворотному порядку, 
а отже --- за допомогою нього можна отримувати 
перестановки лише для наперед заданого $n$. Зауважимо, що кількість
циклів у отриманій перестановці рівна $\sum_{i=1}^n B_i$. 

Розглянемо приклад для $n = 5$. Нехай реалізацією величин $(B_1, B_2, B_3, B_4, B_5)$ є
$(0, 1, 1, 0, 0)$. $B_5 = 0$, тому до першого циклу $(1)$ додається довільно вибране
число з $\left\{2,3,4,5\right\}$ --- наприклад, $3$. Оскільки $B_4 = 0$, то до циклу
$(1, 3)$ додається довільно вибране число з $\left\{2, 4, 5\right\}$ --- наприклад, 4.
Оскільки $B_3 = 1$, то поточний цикл $(1, 3, 4)$ закінчується, а наступний починається з 2.
Нарешті, оскільки $B_2 = 1$, то 5 утворює новий цикл, і отримуємо
перестановку $(1, 3, 4) (2) (5)$.