% !TEX root = ../main.tex
В цьому розділі було доведено низку теорем,
що стосуються граничної поведінки нерухомих точок
випадкової перестановки $\sigma \sim \ESF{n, \theta}$
при $n \to \infty$. Головним результатом є 
доведення грубої збіжності за розподілом 
послідовності точкових процесів, визначених
\eqref{Pn_def}, яка дозволяє досліджувати
розміщення нерухомих точок. Так, застосування
теореми про неперервне відображення \ref{th:cont_map}
дає можливість дослідити граничні розподіли будь-яких
випадкових величин, пов'язаних з нерухомими точками, 
які можна записати як
неперервні функції від декількох дійсних змінних 
(хоча б на одиничному кубі).
Як приклад, було розглянуто розподіли найменшої та найбільшої
нерухомих точок, суми нерухомих точок, найменшого та найбільшого
спейсингів між ними.