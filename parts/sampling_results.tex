% !TEX root = ../main.tex
Для перевірки результатів леми \ref{main_lemma} та теорем
\ref{th:min_max_limit}, \ref{th:sum_limit}, \ref{th:spacing_limit}
скористаємося процесом китайського ресторану для отримання вибірок
з $\ESF{n, \theta}$. Розмір вибірки $m$ в усіх випадках буде рівний 3000.

\subsection{Розподіл кількості нерухомих точок}
Для демонстрації збіжностей
$\P{X_n = k} \to \P{X = k}$, де $X \sim \Poiss{\theta}$,
а $X_n$ визначено в лемі \ref{main_lemma} при $\gamma = 1$,
порівняємо полігони розподілу для $X_n$ при $n = 50, 100, 500$ для
різних значень $\theta$. Ймовірності $\P{X_n = k}$ отримуватимемо наближено
за допомогою закону великих чисел: 
\begin{gather*}
    \P{X_n = k} \approx \frac{1}{m} \sum_{i=1}^m \mathds{1}\left\{ 
        \card\left\{j\in \left\{1,\dots,n\right\} : \sigma_i(j) = j\right\} = k
    \right\},
\end{gather*}
де $\sigma_i \sim \ESF{n, \theta}$ і незалежні.

\makeatletter
    \@for\t:={0.5,1,2,5}\do{
        \begin{figure}[H]
            \centering
            \includegraphics[scale=0.6]{plots/fp_prob_theta_\t.png}
            \caption{Полігони розподілу $X_n$ та $X$ для $\theta = \t$.}
        \end{figure}
    }
\makeatother

Видно, що збіжність ймовірностей дійсно присутня, але для більших
значень $\theta$ вона є повільнішою.

\subsection{Розподіл найменшої та найбільшої нерухомих точок}
Для демонстрації збіжностей
$\frac{\widehat{m}_n}{n} \overset{d}{\longrightarrow} \widehat{m}$ і
$\frac{\widehat{M}_n}{n} \overset{d}{\longrightarrow} \widehat{M}$,
доведених в теоремі \ref{th:min_max_limit}, порівняємо
гістограми $\frac{\widehat{m}_n}{n}$ та $\frac{\widehat{M}_n}{n}$
для $n = 500$ з щільностями розподілу $\widehat{m}$ та $\widehat{M}$
для різних значень $\theta$.

\makeatletter
    \@for\t:={0.5,1,2,5}\do{
        \begin{figure}[H]
            \centering
            \includegraphics[scale=0.6]{plots/fp_min_theta_\t.png}
            \caption{Гістограма $\frac{\widehat{m}_n}{n}$ та щільність $\widehat{m}$ для $\theta = \t$.}
        \end{figure}
    }
\makeatother

\makeatletter
    \@for\t:={0.5,1,2,5}\do{
        \begin{figure}[H]
            \centering
            \includegraphics[scale=0.6]{plots/fp_max_theta_\t.png}
            \caption{Гістограма $\frac{\widehat{M}_n}{n}$ та щільність $\widehat{M}$ для $\theta = \t$.}
        \end{figure}
    }
\makeatother

\subsection{Розподіл суми нерухомих точок}
Для демонстрації збіжності
$\frac{\widehat{S}_n}{n} \overset{d}{\longrightarrow} \widehat{S}$
доведеної в теоремі \ref{th:sum_limit}, порівняємо
гістограми $\frac{\widehat{S}_n}{n}$
для $n = 500$ з щільностями розподілу $\widehat{S}$
для різних значень $\theta$.

\makeatletter
    \@for\t:={0.5,1,2,5}\do{
        \begin{figure}[H]
            \centering
            \includegraphics[scale=0.6]{plots/fp_sum_theta_\t.png}
            \caption{Гістограма $\frac{\widehat{S}_n}{n}$ та щільність $\widehat{S}$ для $\theta = \t$.}
        \end{figure}
    }
\makeatother

\subsection{Розподіл найменшого і найбільшого спейсингів}
Для демонстрації збіжностей
$\frac{\delta_n}{n} \overset{d}{\longrightarrow} \delta$ і
$\frac{\Delta_n}{n} \overset{d}{\longrightarrow} \Delta$,
доведених в теоремі \ref{th:spacing_limit}, порівняємо
емпіричні функції розподілу $\frac{\delta_n}{n}$ та $\frac{\Delta_n}{n}$
при $n = 50, 100, 500$ з емпіричними функціями $\delta$ та $\Delta$
для різних значень $\theta$.

Емпіричну функцію розподілу випадкової величини $X$ за вибіркою
$X_1, ..., X_N$ тут визначено як 
$F^*_X(x) = \frac{1}{m} \sum_{i=0}^{m} \mathds{1}\left\{X_i \leq x\right\}$.

Оскільки для $x \geq 1$ 
$\P{\frac{\delta_n}{n} \leq x} = \P{\frac{\Delta_n}{n} \leq x} = \P{\delta \leq x} = \P{\delta \leq x} = 1$,
то на всіх рисунках емпіричні функції розподілу зображено лише для $x \in [0, 1)$.

\makeatletter
    \@for\t:={0.5,1,2,5}\do{
        \begin{figure}[H]
            \centering
            \includegraphics[scale=0.6]{plots/fp_spacing_min_theta_\t.png}
            \caption{Емпіричні функції розподілу $\frac{\delta_n}{n}$ та $\delta$ для $\theta = \t$.}
        \end{figure}
    }
\makeatother

\makeatletter
    \@for\t:={0.5,1,2,5}\do{
        \begin{figure}[H]
            \centering
            \includegraphics[scale=0.6]{plots/fp_spacing_max_theta_\t.png}
            \caption{Емпіричні функції розподілу $\frac{\Delta_n}{n}$ та $\Delta$ для $\theta = \t$.}
        \end{figure}
    }
\makeatother