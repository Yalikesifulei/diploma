% !TEX root = ../main.tex
\begin{definition}[\cite{Spectorsky}, ст. 114]
    \emph{Перестановкою} $\pi$ на множині $A = \left\{1,\dots,n\right\}$
    називають довільне бієктивне відображення $\sigma: A \to A$.
\end{definition}
\begin{definition}[\cite{Spectorsky}, ст. 118]
    \emph{Циклом довжини $k$} $\left(i_1, ..., i_k\right)$ називають перестановку $\pi$, що змінює
    (зсуває за циклом) елементи $i_1, i_2, \dots, i_k \in A$, залишаючи
    інші на місці, тобто $\pi(i_{j}) = i_{j+1}$ для $j=1,\dots,k-1$,
    $\pi(i_k) = i_1$, $\pi(i_j) = i_j$ для $j = k+1, \dots, n$. 
\end{definition}
\begin{definition}
    Цикли $\left(i_1, ..., i_{k_1}\right)$ та $\left(j_1, ..., j_{k_2}\right)$ 
    на $\left\{1,\dots,n\right\}$ називають незалежними, 
    якщо вони зсувають різні елементи, тобто
    $i_{m_1} \neq j_{m_2}$ для всіх $m_1 = 1,...,k_1$, $m_2 = 1,...,k_2$.
    Незалежні цикли комутують за операцією композиції.
\end{definition}
\begin{theorem}[\cite{Spectorsky}, ст. 119]\label{th:perm_decomposition}
    Кожну перестановку можна зобразити як композицію
    незалежних циклів. Це зображення є єдиним з точністю до
    порядку циклів.
\end{theorem}
\begin{definition}[\cite{Spectorsky}, ст. 116]
    \emph{Групою перестановок (симетричною групою) степеня $n$}
    називають групу, утворену множиною перестановок
    множини $\{1, \dots, n\}$ за операцією композиції.
    Група $\Sym{n}$ містить $n!$ різних перестановок, нейтральним елементом є
    тотожне відображення (\cite{Spectorsky}, ст. 114).
\end{definition}