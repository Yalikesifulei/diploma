% !TEX root = ../main.tex
\begin{definition}[\cite{Spectorsky}, ст. 114]
    \emph{Перестановкою} $\pi$ на скінченній множині $A$
    називають довільне бієктивне відображення $\sigma: A \to A$.
\end{definition}
\begin{definition}[\cite{Spectorsky}, ст. 118]
    \emph{Циклом довжини $k$} називають перестановку $\pi$, що змінює
    (зсуває за циклом) елементи $i_1, i_2, \dots, i_k \in A$, залишаючи
    інші на місці, тобто $\pi(i_{j}) = i_{j+1}$ для $j=1,\dots,k-1$,
    $\pi(i_k) = i_1$, $\pi(i_j) = i_j$ для $j = k+1, \dots, n$. 
\end{definition}
\begin{definition}[\cite{Spectorsky}, ст. 116]
    \emph{Групою перестановок (симетричною групою) степеня $n$}
    називають групу, утворену множиною перестановок
    множини $\{1, \dots, n\}$ за операцією композиції.
    Група $S_n$ містить $n!$ різних перестановок, нейтральним елементом є
    тотожне відображення (\cite{Spectorsky}, ст. 114).
\end{definition}