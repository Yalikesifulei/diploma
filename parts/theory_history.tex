% !TEX root = ../main.tex
Перестановки --- бієктивні відображення на множині
$\left\{1, ..., n\right\}$ --- є одним з найважливіших об'єктів, 
що досліджує комбінаторика.
Добре відомими є їх властивості, як-от
утворення групи $\Sym{n}$ за операцією композиції, можливість розкладу 
на композицію циклів або транспозицій, поняття інверсій та парності (\cite{Spectorsky}, ст. 113-132).

Перестановки можна розглядати і з імовірнісної точки зору,
задавши деякий розподіл (ймовірнісну міру) $\mathbb{P}$ на множині всіх
перестановок $\Sym{n}$ для фіксованого $n \in \N$ і розглядаючи
розподіл випадкових величин, пов'язаних з ними. Прикладом такої
величини є кількість нерухомих точок --- таких 
$i \in \left\{1,...,n\right\}$, які перестановка переводить в себе ж.
Розподіл на $\Sym{n}$, як і будь-яку міру на скінченній множині,
можна задати, поставивши у відповідність кожній з $n!$ перестановок
невід'ємне дійсне число таким чином, щоб їх сума дорівнювала 1.
Найпростішим прикладом є рівномірний розподіл, для якого
$\P{\left\{ \pi \right\}} = \frac{1}{n!}$ для всіх $\pi \in \Sym{n}$.

У даній роботі розглядається однопараметричний розподіл Юенса, 
названий на честь Воррена Юенса --- австрало-американського математика,
який працював з математичною теорією генетики популяцій 
та у 1972 році опублікував у статті <<The sampling theory of selectively alleles>> \cite{EWENS197287}
дослідження розподілу кількості алелів у вибірці генів з великої популяції.
У біології алелі --- це парні гени, що визначають взаємовиключні варіанти
однієї генетичної ознаки (наприклад, високий чи низький зріст, темне або світле волосся тощо).

Юенсом було отримано сумісний розподіл величин 
$\left(C_j(n), 1 \leq j \leq n\right)$ ---
кількості алелів, що зустрінуться у вибірці з $n$ генів $j$ разів:
\begin{gather}\label{ESF_original}
    \P{
        C_1(n) = c_1, C_2(n) = c_2, ..., C_n(n) = c_n
    } = 
    \frac{n!}{\theta_{(n)}} 
    \prod_{j=1}^n \left(\frac{\theta}{j}\right)^{c_j} \frac{1}{c_j!},
\end{gather}
де $\sum_{j=1}^n j c_j = n$, $\theta_{(n)} = \theta (\theta + 1) \dots (\theta + n - 1)$,
а параметр $\theta > 0$ пов'язаний зі швидкістю появи нових типів алелів у популяції.
Історію появи формули \eqref{ESF_original}, що відіграла значну роль не тільки
в генетиці популяцій, а й теорії випадкових відображень та перестановок,
детально викладено у статі Саймона Таваре <<The magical Ewens sampling formula>> \cite{Tavare},
виданої на честь 50-тої річниці написання відповідної статті Юенса.

Розглянемо однопараметричний розподіл на $\Sym{n}$, для якого
$\P{\left\{ \pi \right\}} = C(\theta) \cdot \theta^{\cycle(\pi)}$, тобто
ймовірність вибору перестановки пропорційна
$\theta^{\cycle(\pi)}$, де $\theta > 0$ і $\cycle(\pi)$ позначає кількість циклів у
розкладі перестановки $\pi$ в композицію незалежних циклів, єдиність якої гарантується теоремою \ref{th:perm_decomposition}.
Відомо (\cite{Abramowitz_Stegun}, ст. 824), що кількість
перестановок в $\Sym{n}$ з $k$ циклами дорівнює 
$\left[{n\atop k}\right]$ --- модулю числа Стірлінґа першого роду. Щоб знайти
константу нормування $C(\theta)$, скористаємося тим, що
$\P{\Sym{n}} = 1$:
\begin{gather*}
    \P{\Sym{n}} = \sum_{\pi \in \Sym{n}} \P{\left\{ \pi \right\}} = 
    C(\theta) \sum_{\pi \in \Sym{n}} \theta^{\cycle(\pi)} = 
    C(\theta) \sum_{k=1}^n \left[{n\atop k}\right] \theta^k = 1
\end{gather*}
З властивостей чисел Стірлінґа першого роду
$\sum_{k=1}^n \left[{n\atop k}\right] \theta^k = \theta_{(n)}$, тому
$C(\theta) = \frac{1}{\theta_{(n)}}$.
\begin{definition}
    \emph{Розподілом Юенса} на групі перестановок $\Sym{n}$ з параметром $\theta > 0$
    називається розподіл, для якого
    \begin{equation}\label{ESF}
        \mathbb{P}(\{\pi\}) = \frac{
            \theta^{\cycle(\pi)}
        }{
            \theta (\theta + 1) \dots (\theta + n - 1)
        }, \; \pi \in \Sym{n},
    \end{equation}
    де $\cycle(\pi)$ позначає кількість циклів у $\pi$.
\end{definition}
Тут і далі
відповідні випадкові перестановки називатимемо
\emph{перестановками Юенса} і, за потреби, для позначення
такої перестановки $\sigma$ на $\Sym{n}$ застосовуватимемо
позначення $\sigma \sim \ESF{n, \theta}$.
Якщо $\theta = 1$, то формула \eqref{ESF} задає рівномірний розподіл на $\Sym{n}$.

Встановимо відповідність між \eqref{ESF_original} та \eqref{ESF}.
Нехай $\sigma \sim \ESF{n, \theta}$, а $\cycle_j(\sigma)$ позначає кількість
циклів довжини $j$ у розкладі $\sigma$ в композицію незалежних циклів.
Знайдемо $\P{\cycle_1(\sigma) = c_1,...,\cycle_n(\sigma) = c_n}$, 
де $\sum_{j=1}j c_j = n$:
\begin{gather*}
    \P{\cycle_1(\sigma) = c_1,...,\cycle_n(\sigma) = c_n} =
    \sum_{
        \substack{\pi \in \Sym{n} :\; \forall j = 1,...,n \\ \cycle_j(\pi) = c_j}
    } \mathbb{P}(\{\pi\}) = 
    \frac{1}{\theta_{(n)}} \sum_{
        \substack{\pi \in \Sym{n} :\; \forall j = 1,...,n \\ \cycle_j(\pi) = c_j}
    } \theta^{\cycle(\pi)} = \\ =
    \frac{1}{\theta_{(n)}} \sum_{
        \substack{\pi \in \Sym{n} :\; \forall j = 1,...,n \\ \cycle_j(\pi) = c_j}
    } \theta^{c_1 + ... + c_n} = 
    \frac{\theta^{c_1 + ... + c_n}}{\theta_{(n)}}
    \sum_{\pi \in \Sym{n}} \mathds{1}\left\{\forall \; j = 1,...,n : \cycle_j(\pi) = c_j\right\}.
\end{gather*}
З \cite{LogStructures}, ст. 11, відомо, що кількість
перестановок $\pi$, для яких $\cycle_j(\pi) = c_j, j = 1,...,n$,
дорівнює $n! \prod_{j=1}^n \left(\frac{1}{j}\right)^{c_j} \frac{1}{c_j!}$, 
тому
\begin{gather*}
    \P{\cycle_1(\sigma) = c_1,...,\cycle_n(\sigma) = c_n} =
    \frac{\theta^{c_1 + ... + c_n}}{\theta_{(n)}} \cdot n! \prod_{j=1}^n \left(\frac{1}{j}\right)^{c_j} \frac{1}{c_j!} =
    \frac{n!}{\theta_{(n)}} 
    \prod_{j=1}^n \left(\frac{\theta}{j}\right)^{c_j} \frac{1}{c_j!}.
\end{gather*}
Отже, розподіл кількості циклів різної довжини у перестановці з розподілом \eqref{ESF} збігається 
з розподілом кількості алелів у вибірці з $n$ генів, заданим \eqref{ESF_original}.