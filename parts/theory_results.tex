% !TEX root = ../main.tex
Як було сказано раніше, формула \eqref{ESF_original}
виникла під час дослідження вибірок генів.
В реальних задачах $n$ може бути досить великим,
тому постає питання наближеного обчислення відповідних ймовірностей.
Так, добре дослідженими є граничні розподіли кількості циклів
у випадкових перестановках з розподілом \eqref{ESF}.
Зауважимо, що кількість нерухомих точок --- це кількість
циклів довжини 1.

Оскільки найпростішим випадком формули \eqref{ESF} є випадок рівномірного
розподілу при $\theta = 1$, почнемо огляд наявних результатів з нього.
Найвідомішим є так звана <<задача про неуважну секретарку>>,
в якій необхідно знайти ймовірність того, що з $n$ листів з
навмання написаними адресами $k$ надійдуть за призначенням.
В термінах випадкової перестановки з розподілом
$\ESF{n, 1}$ йдеться саме про кількість нерухомих точок. Хоча відповідні
ймовірності дуже просто записати у явному вигляді, перейдемо одразу
до більш складного результату:
\begin{theorem}[\cite{LogStructures}, ст. 12]
    Для $\sigma_n \sim \ESF{n, 1}$ та $j = 1,...,n$ 
    \begin{gather}\label{cycles_distr_uniform}
        \P{\cycle_j(\sigma_n) = k} = \frac{j^{-k}}{k!} \sum_{i=0}^{
            \floor*{n/j} - k
        } (-1)^i \frac{j^{-i}}{i!},
    \end{gather}
    де $\cycle_j(\sigma_n)$ позначає кількість
    циклів довжини $j$ у $\sigma_n$.
\end{theorem}
З формули \eqref{cycles_distr_uniform} видно,
що 
\begin{gather}\label{cycles_limit_uniform_pointwise}
    \lim_{n\to\infty} \P{\cycle_j(\sigma_n) = k} = 
    \frac{j^{-k}}{k!} \sum_{i=0}^{\infty} (-1)^i \frac{j^{-i}}{i!} = 
    \frac{j^{-k}}{k!} e^{-1/j},
\end{gather}
тобто граничним розподілом кількості
циклів довжини $j$ у
рівномірно випадковій перестановці на $\Sym{n}$
є $\Poiss{1/j}$.
Зокрема, в задачі про неуважну секретарку
при великих $n$ ймовірність того, що за призначенням
надійде $k$ листів, приблизно дорівнює
$\frac{1}{k!} e^{-1}$. 

Виявляється, що випадкові величини $X_j \sim \Poiss{1/j}$,
які в силу \eqref{cycles_limit_uniform_pointwise}
є граничними для $\cycle_j(\sigma_n)$ при $n\to\infty$,
є незалежними:

\begin{theorem}[\cite{LogStructures}, ст. 14]\label{cycles_limit_uniform_joint}
    Нехай $\sigma_n \sim \ESF{n, 1}$. Тоді
    для будь-якого $k \in \N$
    випадковий вектор 
    $\left(\cycle_{1}(\sigma_n), ..., \cycle_{k}(\sigma_n)\right)^T$
    збігається за розподілом до
    $\left(X_{1}, ..., X_{k}\right)^T$,
    де $X_{j} \sim \Poiss{1/{j}}$ і 
    всі $X_{1}, ..., X_{k}$ є незалежними.
\end{theorem}

Теорема \ref{cycles_limit_uniform_joint} узагальнюється на
випадок $\sigma_n \sim \ESF{n, \theta}$ для довільних $\theta > 0$.
\begin{theorem}[\cite{Arratia}, ст. 520]\label{cycles_limit_ewens_joint}
    Нехай $\sigma_n \sim \ESF{n, \theta}$. Тоді
    для будь-якого $k \in \N$
    випадковий вектор 
    $\left(\cycle_{1}(\sigma_n), ..., \cycle_{k}(\sigma_n)\right)^T$
    збігається за розподілом до
    $\left(X_{1}, ..., X_{k}\right)^T$,
    де $X_{j} \sim \Poiss{\theta/{j}}$ і 
    всі $X_{1}, ..., X_{k}$ є незалежними.
\end{theorem}

Також варто навести рівність (\cite{LogStructures}, ст. 59), яка пов'язує
розподіли випадкових векторів $\left(\cycle_{1}(\sigma_n), ..., \cycle_{n}(\sigma_n)\right)^T$
та $\left(X_1, ..., X_n\right)^T$ з теореми \ref{cycles_limit_ewens_joint}:
\begin{gather}
    \P{
        \cycle_{1}(\sigma_n) = c_1, \cycle_{2}(\sigma_n) = c_2, ..., \cycle_{n}(\sigma_n) = c_n
    } = \\ =
    \P{
        X_1 = c_1, X_2 = c_2, ..., X_n = c_n \mid
        \sum_{j=1}^n j X_j = n
    },
\end{gather} 
де $c_1, ..., c_n \in \N_0$ і
$\sum_{j=1}^n j c_j = n$.