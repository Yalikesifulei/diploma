% !TEX root = ../main.tex
\begin{definition}[\cite{Kallenberg_2017}, ст. 19]
    Для будь-якого простору $X$
    непорожня сім'я підмножин $\mathcal{R}$ 
    називається \emph{кільцем}, якщо
    вона замкнена відносно скінченних об'єднань, перетинів
    та різниць. Еквівалентне означення (\cite{Berezanskij}, ст. 4):
    сім'я $\mathcal{R}$ непорожня та
    замкнена відносно скінченних об'єднань та різниць.
    % $\left(A, B \in \mathcal{R}\right) \Rightarrow \left(A \cup B, A \setminus B\in \mathcal{R}\right)$.
\end{definition}
\begin{definition}[\cite{Kallenberg_2017}, ст. 19]
    Для будь-якого простору $X$
    непорожня сім'я підмножин $\mathcal{S}$ 
    називається \emph{напівкільцем}, якщо
    вона замкнена відносно скінченних перетинів та кожна різниця
    множин з $\mathcal{S}$ представляється
    у вигляді диз'юнктного об'єднання множин з $\mathcal{S}$, тобто
    для будь-яких $A, B \in \mathcal{S}$ існують
    множини $K_i \in \mathcal{S}, i = 1,\dots,n$,
    що попарно не перетинаються і $A \setminus B = \bigcup_{i=1}^n K_i$.
\end{definition}
\begin{definition}[\cite{Bog}, ст. 139]
    Для будь-якого простору $X$
    непорожня сім'я підмножин $\mathcal{A}$ 
    називається \emph{$\sigma$-алгеброю},
    якщо виконуються наступні три умови:
    \begin{enumerate}
        \item $\left(A \in \mathcal{A}\right) \Rightarrow \left(A^C = X \setminus A \in \mathcal{A}\right)$;
        % \item $\left(A, B \in \mathcal{A}\right) \Rightarrow \left(A \cup B\in \mathcal{A}\right)$; 
        \item $\left(A_1, A_2, A_3, ... \in \mathcal{A}\right) \Rightarrow \left(\bigcup_{n=1}^{\infty} A_n \in \mathcal{A}\right)$.
    \end{enumerate} 
    Пара $\left(X, \mathcal{A}\right)$ називається \emph{вимірним простором}.
\end{definition}
\begin{definition}[\cite{Bog}, ст. 146]
    Нехай $\left(X, \mathcal{A}_X\right)$ та $\left(Y, \mathcal{A}_Y\right)$ ---
    два вимірних простори. Відображення $f: X \to Y$ називається 
    \emph{вимірним}, якщо для кожної множини
    $A \in \mathcal{A}_Y$ її повний прообраз
    $f^{-1}(A) = \left\{x : f(x) \in A\right\}$
    належить $\mathcal{A}_X$.
\end{definition}
\begin{definition}[\cite{Bog}, ст. 147]
    Нехай $X$ --- метричний простір, $\mathcal{O}$ ---
    сім'я всіх відкритих підмножин $X$. Мінімальна $\sigma$-алгебра
    $\mathcal{B}(X)$, що містить $\mathcal{O}$, називається
    \emph{борелевою $\sigma$-алгеброю}, а множини
    $A \in \mathcal{B}(X)$ --- \emph{борелевими множинами.}
\end{definition}
\begin{definition}[\cite{Kallenberg_2017}, ст. 24]
    Сім'я підмножин $\mathcal{S}$ сепарабельного метричного простору $X$ називається
    \emph{розсікаючою}, якщо виконуються наступні дві умови:
    \begin{enumerate}
        \item Кожну відкриту підмножину $X$ можна зобразити як зліченне об'єднання множин з $\mathcal{S}$;
        \item Кожну обмежену підмножину $X$ можна покрити скінченною кількістю множин з $\mathcal{S}$.
    \end{enumerate}
    Для простору $\R^n$ прикладом розсікаючої сім'ї множин є сім'я куль з
    раціональними радіусами та центрами в точках з раціональними координатами.
\end{definition}
\begin{definition}[\cite{Berezanskij}, ст. 8]
    Нехай $\mathcal{A}$ --- $\sigma$-алгебра
    у просторі $X$. Функція $\mu: \mathcal{A} \to \R$ називається
    \emph{мірою} на вимірному просторі $\left(X, \mathcal{A}\right)$, якщо виконуються наступні дві умови:
    \begin{enumerate}
        \item Невід'ємність: $\forall \; A \in \mathcal{A} : \mu(A) \geq 0$;
        \item $\sigma$-адитивність: для довільних множин $A_1, A_2, A_3, ... \in \mathcal{A}$,
        що попарно не перетинаються, 
        $\mu\left(\bigcup_{n=1}^{\infty} A_n\right) = \sum_{n=1}^{\infty}\mu(A_n)$.
    \end{enumerate}
\end{definition}
\begin{definition}
    Міра $\mu$ на вимірному просторі $\left(X, \mathcal{A}\right)$ називається
    \emph{ймовірнісною}, якщо $\mu(X) = 1$.
\end{definition}
\begin{definition}[\cite{Kallenberg_2017}, ст. 22]
    Нехай $\left(X, \mathcal{A}\right)$ --- вимірний простір, для якого $\{ x \} \in \mathcal{A}$
    для всіх $x \in X$.
    Точка $x \in X$ називається \emph{атомом} міри $\mu$ на
    $\left(X, \mathcal{A}\right)$, якщо $\mu\left(\{x \}\right) > 0$.
\end{definition}
\begin{definition}[\cite{Kallenberg_2017}, ст. 22; \cite{Resnick_1987}, ст. 123]
    \emph{Міра Дірака}, зосереджена
    в точці $x \in X$ --- це міра $\delta_x$ на 
    на вимірному просторі $\left(X, \mathcal{A}\right)$,
    для якої $\forall A \in \mathcal{A}: \delta_x(A) = \mathds{1}\left\{x \in A\right\} = 
    \begin{cases}
        1, & x \in A, \\
        0, & x \notin A.
    \end{cases}$
\end{definition}
\begin{definition}[\cite{Resnick_1987}, ст. 123]
    \emph{Точкова міра} --- це міра $\mu$ на 
    на вимірному просторі $\left(X, \mathcal{A}\right)$,
    для якої $\forall A \in \mathcal{A}: \mu(A) = \sum_{i=1}^{\infty} \delta_{x_i}(A)$,
    де $\left(x_i, i \geq 1\right)$ --- зліченний набір точок $X$, не обов'язково різних.
    У випадку, коли $X$ --- метричний простір, точкова міра називається \emph{радоновою},
    якщо міра компактних множин з $\mathcal{A}$ завжди є скінченною.
\end{definition}
\begin{definition}[\cite{Resnick_1987}, ст. 124]
    Точкова міра $\mu$ на вимірному просторі $\left(X, \mathcal{A}\right)$ називається \emph{простою},
    якщо для всіх $x \in X$ $\mu\left(\{x\}\right) \leq 1$.
\end{definition}
% \begin{definition}[\cite{Bog}, ст. 147]
%     Нехай $\left(X, \mathcal{A}\right)$ --- вимірний простір. Відображення
%     $f: X \to \R$ називається \emph{вимірною функцією}, якщо 
%     $\forall \; A \in \mathcal{A}: f^{-1}(A) \in \mathcal{B}(\R)$.
% \end{definition}
% \begin{definition}[\cite{Bog}, ст. 152]
%     Нехай $\left(f_n, n\geq1\right)$ --- послідовність вимірних функцій
%     на вимірному просторі
%     $\left(X, \mathcal{A}\right)$, а $\mu$ --- міра на цьому просторі.
%     Послідовність $\left(f_n, n\geq1\right)$ збігається до вимірної функції $f$
%     на тому ж просторі, якщо
%     $$\mu \left(
%         \left\{
%             x : \lim_{n\to\infty} f_n(x) \neq f(x)
%         \right\}
%     \right) = 0.$$
%     Ця збіжність позначається $f_n \overset{\text{м.в.}}{\longrightarrow} f$.
% \end{definition}
% \begin{definition}[\cite{Bog}, ст. 152]
%     Нехай $\left(f_n, n\geq1\right)$ --- послідовність вимірних функцій
%     на вимірному просторі
%     $\left(X, \mathcal{A}\right)$, а $\mu$ --- міра на цьому просторі.
%     Послідовність $\left(f_n, n\geq1\right)$ \emph{збігається за мірою} до вимірної функції $f$,
%     якщо
%     $$\forall \; \varepsilon > 0 : \mu \left(
%         \left\{
%             x : \left|f_n(x) - f(x)\right| \geq \varepsilon
%         \right\}
%     \right) \to 0, \; n \to \infty.$$
%     Ця збіжність позначається $f_n \overset{\mu}{\longrightarrow} f$.
% \end{definition}
\begin{theorem}\label{th:series_dct}
    Нехай $\left(f_n, n \geq 1\right)$ --- послідовність функцій 
    $\N_0 \to \R$, для яких $\sum_{i=0}^{\infty} |f_n(i)| < \infty$. Якщо
    існують такі $f, g: \N_0 \to \R$, що
    для всіх $i$ $f_n(i) \to f(i)$, $|f_n(i)| \leq g(i)$ і 
    $\sum_{i=0}^{\infty} g(i) < \infty$, то
    \begin{enumerate}
        \item $\sum_{i=0}^{\infty} |f(i)| < \infty$;
        \item $\sum_{i=0}^{\infty} f(i) = \lim_{n\to\infty} \sum_{i=0}^{\infty} f_n(i)$;
        \item $\sum_{i=0}^{\infty} \left|f_n(i) - f(i)\right| \to 0, \; n \to \infty$.
    \end{enumerate}
\end{theorem}
Теорема \ref{th:series_dct} є окремим випадком теореми Лебега про мажоровану збіжність
(\cite{Bog}, ст. 164) для міри $\mu$ на $\left(\N_0, 2^{\N_0}\right)$,
визначеної за правилом $\mu(A) = \card(A)$.

\begin{definition}[\cite{Resnick_1987}, ст. 140]
    Нехай $\left(\mu_n, n \geq 1\right)$ --- послідовність мір на
    на вимірному просторі $\left(X, \mathcal{A}\right)$,
    де $X$ є метричним простором,
    а $C_K^+(X)$ --- множина неперервних невід'ємних функцій
    $X \to \R$ з компактним носієм.
    Послідовність $\left(\mu_n, n \geq 1\right)$
    \emph{грубо збігається} до міри $\mu$ на тому ж вимірному просторі,
    якщо виконується $\int_X f \d\mu_n \to \int_X f \d\mu$ для всіх $f \in C_K^+(X)$.
    Ця збіжність позначається $\mu_n \overset{v}{\longrightarrow} \mu$.
\end{definition}

Надалі вважатимемо, що якщо мова йде про
грубу збіжність послідовності мір, то простір, на якому вони задані,
є метричним.
Наведемо теорему, що характеризує збіжність послідовності точкових мір.
\begin{theorem}[\cite{Resnick_1987}, ст. 144]\label{th:point_mes_conv}
    Нехай $\left(\mu_n, n \geq 1\right)$ та $\mu$ --- міри
    на вимірному просторі $\left(X, \mathcal{A}\right)$ і
    $\mu_n \overset{v}{\longrightarrow} \mu$. Для кожної компактної множини
    $K \subset X$ з $\mu(\partial K) = 0$ існує номер $N = N(K)$ такий,
    що при $n \geq N$ існують нумерації атомів $\mu_n$ та 
    $\mu$, $x_i^{(n)}, 1 \leq i \leq p$ та $x_i, 1 \leq i \leq p$ відповідно, такі, що
    $$
        \mu_n(A \cap K) = \sum_{i=1}^p \delta_{x_i^{(n)}} (A), \;
        \mu(A \cap K) = \sum_{i=1}^p \delta_{x_i} (A)
    $$
    для всіх $A \in \mathcal{A}$ і $x_i^{(n)} \to x_i$ для всіх $1 \leq i \leq p$.
\end{theorem}

\begin{definition}[\cite{billingsley}, ст.121-124]
    \emph{Простором càdlàg-функцій} на $[0, 1]$ називається простір $\mathcal{D}_{[0, 1]}$
    функцій $f : [0, 1] \to \R$, які неперервні справа і мають границі зліва.
\end{definition}

\begin{definition}
    \emph{Метрикою Скорохода} на $\mathcal{D}_{[0, 1]}$ називається метрика, визначена за формулою
    \begin{gather*}
        d(f, g) = \inf_{\lambda \in \Lambda} \max \left(
            \sup_{x \in [0, 1]} \left|\lambda(x) - x\right|, 
            \sup_{x \in [0, 1]} \left|f(x) - g(\lambda(x))\right|
        \right),
    \end{gather*}
    де $\Lambda$ --- множина строго зростаючих неперервних відображень
    $[0, 1]$ в себе.

    Послідовність функцій $f_n \in \mathcal{D}_{[0, 1]}$ збігається за метрикою Скорохода до $f \in \mathcal{D}_{[0, 1]}$
    тоді і тільки тоді, коли існує послідовність функцій $\lambda_n \in \Lambda$ таких, що
    рівномірно відносно $x$
    $\lim_{n\to\infty} f_n \left(\lambda_n(x)\right) = f(x)$ та
    $\lim_{n\to\infty} \lambda_n(x) = x$, тобто
    виконуються граничні співвідношення
    \begin{gather*}
        \lim_{n\to\infty} \sup_{x\in[0, 1]}\left| f_n \left(\lambda_n(x)\right) - f(x) \right| = 0, \;
        \lim_{n\to\infty} \sup_{x\in[0, 1]}\left| \lambda_n(x) - x \right| = 0.
    \end{gather*}
\end{definition}

% \begin{definition}[\cite{prob_metrics}]
%     \emph{Метрикою Колмогорова} для ймовірнісних мір на вимірному просторі
%     $\left(\R, \mathcal{B}(\R)\right)$
%     називається метрика, визначена за формулою
%     \begin{gather}
%         d_K(\mu, \nu) = \sup_{x \in \R} \left|
%             \mu\left((-\infty, x]\right) - \nu\left((-\infty, x]\right)
%         \right|.
%     \end{gather}
% \end{definition}

% \begin{definition}[\cite{prob_metrics}]
%     \emph{Метрикою повної варіації} для ймовірнісних мір на 
%     вимірному просторі $\left(X, \mathcal{A}\right)$
%     називається метрика, визначена за формулою
%     \begin{gather}
%         d_{TV}(\mu, \nu) = \sup_{A \in \mathcal{A}} \left|
%             \mu\left(A\right) - \nu\left(A\right)
%         \right|.
%     \end{gather}
%     Якщо $X$ є зліченним, то 
%     \begin{gather}
%         d_{TV}(\mu, \nu) = \frac{1}{2} \sum_{x \in X} \left|\mu(\{x\}) - \nu(\{x\})\right|.
%     \end{gather}
% \end{definition}

% Варто зазначити, що для ймовірнісних мір на $\R$
% має місце $d_K(\mu, \nu) \leq d_{TV}(\mu, \nu)$,
% оскільки $\left\{(-\infty, x], x \in \R \right\} \subset \mathcal{B}(\R)$.