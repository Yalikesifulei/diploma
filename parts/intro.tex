% !TEX root = ../main.tex
\hspace{\parindent}
В даній роботі розглядаються граничні теореми для
випадкових перестановок на симетричній групі $\Sym{n}$
при $n\to\infty$, що мають однопараметричний розподіл Юенса,
згідно з яким ймовірність отримати перестановку
$\pi$ пропорційна $\theta^{\cycle(\pi)}$, де
$\cycle(\pi)$ позначає кількість незалежних
циклів у розкладі $\pi$, а $\theta > 0$
є параметром розподілу.
Хоча означення таких перестановок вперше
виникло в контексті генетики, робота присвячена
їх дослідженню з математичної точки зору.

Мета роботи полягає у формулюванні та доведенні
низки граничних теорем, що стосуються розподілу
нерухомих точок --- тобто таких, які перестановка
залишає на своєму місці.

Методи дослідження ґрунтуються на поняттях та теоремах
алгебри, теорії міри, функціонального аналізу та теорії випадкових процесів.

Перший розділ присвячено огляду всіх необхідних
для роботи теоретичних відомостей з вищезгаданих дисциплін.

Другий розділ висвітлює історію виникнення
поняття перестановок Юенса, їхнє практичне застосування та
наявні результати, що стосуються їхньої граничної поведінки.

Третій розділ містить власні результати автора з дослідження
граничної поведінки перестановок Юенса і є
найцікавішим та найскладнішим з математичної точки зору.

Четвертий розділ присвячено огляду методів моделювання
перестановок Юенса та дослідженню збіжностей,
доведених в граничних теоремах третього розділу,
шляхом порівняння полігонів розподілу, гістограм
та емпіричних функцій розподілу вибірок, отриманих
внаслідок моделювання, з
полігонами розподілу, щільностями та функціями розподілу
граничних випадкових величин.