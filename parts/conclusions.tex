% !TEX root = ../main.tex
\hspace{\parindent}
В даній дипломній роботі було розглянуто випадкові перестановки Юенса
$\sigma \sim \ESF{n, \theta}$, задані розподілом \eqref{ESF}. 
Через нерухомі точки таких перестановок означено послідовність 
точкових процесів $P_n$ на $[0, 1]$ формулою
\eqref{Pn_def} та доведено збіжність при $n \to \infty$ цієї
послідовності до однорідного точкового процесу
Пуассона з інтенсивністю $\theta$ в двох,
в цьому випадку еквівалентних, сенсах --- грубу
збіжність за розподілом та збіжність в топології Скорохода.

За допомогою теореми про неперервне відображення
\ref{th:cont_map} було доведено граничні теореми для
деяких статистик, пов'язаних з нерухомими точками
перестановок Юенса, а саме --- найменшої та найбільшої точок, 
суми точок і найменшої та найбільшої
відстаней між сусідніми нерухомими точками (спейсингів).
Для відповідних граничних випадкових величин було отримано
розподіли в явному вигляді. Усі ці величини є змішаними, 
оскільки з ненульовою ймовірністю
нерухомих точок може не бути взагалі, тому розглянуто
також їх абсолютно неперервні версії, розподіли яких отримані 
накладанням умови існування таких точок.

Результат теореми \ref{main_th} є найважливішим у роботі,
оскільки в поєднанні з теоремою \ref{th:cont_map} дозволяє 
отримувати граничні теореми для всіх статистик, які можна представити
як неперервні функції від атомів точкової міри. Однак він не є вичерпним
з точки зору дослідження властивостей перестановок Юенса,
оскільки стосується лише нерухомих точок. Напрямком для продовження проведеної
роботи може бути дослідження граничної поведінки циклів у таких перестановках.