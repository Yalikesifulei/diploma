% !TEX root = ../main.tex
$\mathds{1}\{\; \cdot\;\}$ --- індикаторна функція, що дорівнює 1 у випадку, коли умова в
дужках справджується, і 0 у іншому випадку.

$\card X$ --- потужність множини $X$. 

$\ceil*{x}$ --- найменше ціле число, яке більше за або дорівнює дійсному числу $x$.%, $\min\left\{k\in\Z : k \geq x\right\}$.

$\floor*{x}$ --- найбільше ціле число, яке менше за або дорівнює дійсному числу $x$.%, $\max\left\{k\in\Z : k \leq x\right\}$.

$\N_0$ --- множина цілих невід'ємних чисел, $\N_0 = \N \cup \{0\}$.

$\mathrm{S}_n$ --- група перестановок (симетрична група) степеня $n$.

$C_K^+(X)$ --- множина неперервних невід'ємних функцій
$X \to \R$ з компактним носієм.

$\mathcal{B}(X)$ --- борелева $\sigma$-алгебра на множині $X$.

$M_p(E)$ --- множина усіх точкових мір, визначених на просторі $E$.

$\left<a,b\right>$ --- інтервал, позначає одне з $[a, b]$, $(a, b)$, $[a, b)$ чи $(a, b]$.

$\delta_x$ --- міра Дірака, зосереджена в точці $x$.

$\mathrm{Leb}$ --- міра Лебега.

$\mathcal{L}\left\{f\right\}$ --- перетворення Лапласа функції $f$.

$\psi_N$ --- функціонал Лапласа точкового випадкового процесу $N$.

$\limsup_{n\to\infty} a_n$ --- верхня границя послідовності $a_n$.

$a_n \to a$ --- числова послідовність $a_n$ збігається до $a$.

$\mu_n \overset{v}{\longrightarrow} \mu$ --- послідовність мір $\mu_n$
грубо збігається до міри $\mu$.

$\xi_n \overset{vd}{\longrightarrow} \xi$ --- послідовність точкових випадкових процесів $\xi_n$
грубо збігається за розподілом до точкового випадкового процесу $\xi$.

$X_n \overset{Sd}{\longrightarrow} X$ --- послідовність випадкових процесів $\xi_n$
збігається за розподілом у топології Скорохода до випадкового процесу $X$.

$X_n \overset{d}{\longrightarrow} X$ --- послідовність випадкових величин $X_n$
збігається за розподілом до випадкової величини $X$.

$X \overset{d}{=} Y$ --- випадкові величини $X$ та $Y$ рівні за розподілом.

$X_{(k)}$ --- $k$-та порядкова статистика, тобто $k$-та за номером 
випадкова величина серед відсортованих у порядку зростання неперервних
випадкових величин $X_1, ..., X_n$.

$X_{(k)}^{[n]}$ --- $k$-та порядкова статистика для $n$ випадкових величин.

$\E X$ --- математичне сподівання випадкової величини $X$.

$X \sim P$ --- випадкова величина $X$ має розподіл $P$.

$\Poiss{a}$ --- дискретний розподіл Пуассона з параметром $a > 0$, $\P{X = n} = \frac{a^n}{n!}e^{-a}$ для $n \in \N_0$.

$\Unif{a, b}$ --- абсолютно неперервний рівномірний розподіл на інтервалі $\left<a,b\right>$ зі щільністю
$f(x) = \frac{1}{b-a} \cdot \mathds{1}\left\{x \in \left<a,b\right> \right\}$.

$\Exp{\lambda}$ --- абсолютно неперервний експоненційний розподіл з параметром $\lambda > 0$ зі щільністю
$f(x) = \lambda e^{-\lambda x} \cdot \mathds{1}\left\{x \geq 0\right\}$.

$\ESF{n, \theta}$ --- розподіл Юенса на $\Sym{n}$ з параметрами $n \in \N$, $\theta > 0$.

$I_{\nu}(z)$ --- модифікована функція Бесселя першого роду, $\nu \in \R$.