\renewcommand{\P}[1]{\mathbb{P}\left(#1\right)}

\newcommand{\Pn}[1]{\mathbb{P}'\left(#1\right)}
\newcommand{\E}{\mathbb{E}}
\newcommand{\D}{\mathbb{D}}
\newcommand{\N}{\mathbb{N}}
\newcommand{\Z}{\mathbb{Z}}
\newcommand{\R}{\mathbb{R}}
\newcommand{\X}{\mathbb{X}}
\newcommand{\g}{\gamma}
\newcommand{\Poiss}[1]{\mathrm{Pois}\left(#1\right)}
\newcommand{\Unif}[1]{\mathrm{U}\left(#1\right)}
\newcommand{\Exp}[1]{\mathrm{Exp}\left(#1\right)}
\newcommand{\Sym}[1]{\mathrm{S}_{#1}}
\newcommand{\ESF}[1]{\mathrm{ESF}\left(#1\right)}
\DeclareMathOperator{\card}{card}
\DeclarePairedDelimiter\ceil{\lceil}{\rceil}
\DeclarePairedDelimiter\floor{\lfloor}{\rfloor}
\DeclareMathOperator{\Sum}{sum}
\renewcommand{\L}[1]{\mathcal{L}\left\{#1\right\}}
\DeclareMathOperator{\smax}{s-max}
\DeclareMathOperator{\smin}{s-min}
\DeclareMathOperator{\cycle}{c}
\DeclareMathOperator{\inv}{i}
\newcommand{\cov}[2]{\mathrm{cov}\left(#1, #2\right)}

\renewcommand{\d}{\mathrm{d}}

\newcommand*{\defeq}{\stackrel{\text{def}}{=}}
\makeatletter
\newcommand\incircbin
{%
  \mathpalette\@incircbin
}
\newcommand\@incircbin[2]
{%
  \mathbin%
  {%
    \ooalign{\hidewidth$#1#2$\hidewidth\crcr$#1\bigcirc$}%
  }%
}
\newcommand{\oeq}{\incircbin{=}}
\makeatother

\newtheorem{theorem}{Теорема}[section]
\newtheorem*{theorem*}{Теорема}
\newtheorem{corollary}{Наслідок}[theorem]
\newtheorem{lemma}[theorem]{Лема}
\newtheorem*{remark}{Зауваження}

\theoremstyle{definition} % "Определение"
\newtheorem{definition}{Означення}[section]
\newtheorem*{definition*}{Означення}