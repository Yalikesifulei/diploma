% !TEX program = xelatex
% https://habr.com/ru/post/144648/
\documentclass[a4paper,14pt]{extreport}
\usepackage{xltxtra}

\usepackage{extsizes}
%\usepackage{cmap} % для кодировки шрифтов в pdf
\usepackage{fontspec}
\defaultfontfeatures{Ligatures={TeX}} 
\setmainfont[Ligatures=TeX]{Times New Roman}
\setmonofont[Mapping=tex-text]{Courier New}
\newfontfamily\cyrillicfonttt{Courier New}
\usepackage{polyglossia}
\setdefaultlanguage{ukrainian}
\setotherlanguages{english}
% \usepackage[english, main=ukrainian]{babel}

\usepackage{graphicx} % для вставки картинок
\usepackage{amssymb,amsfonts,amsmath,amsthm,mathtools} % математические дополнения от АМС
\usepackage{indentfirst} % отделять первую строку раздела абзацным отступом тоже
\usepackage[usenames,dvipsnames]{color} % названия цветов
\usepackage{makecell}
\usepackage{multirow} % улучшенное форматирование таблиц
\usepackage{ulem} % подчеркивания

\usepackage{tikz}
\usepackage{pgfplots}
\pgfplotsset{compat=1.17}
\usetikzlibrary{positioning,arrows.meta,shapes}
% \usepackage[europeanresistors, RPvoltages]{circuitikz}
% \usetikzlibrary{arrows,decorations.markings}

\usepackage[nodisplayskipstretch]{setspace}
\onehalfspacing % полуторный интервал
\frenchspacing

\usepackage{fancyhdr}
\pagestyle{fancy}
\fancyhf{}
\fancyhead[R]{\thepage}
\fancyheadoffset{0mm}
\fancyfootoffset{0mm}
\setlength{\headheight}{17pt}
\renewcommand{\headrulewidth}{0pt}
\renewcommand{\footrulewidth}{0pt}
\fancypagestyle{plain}{ 
    \fancyhf{}
    \rhead{\thepage}}
\setcounter{page}{1} 

\usepackage[tableposition=top]{caption}
\usepackage{subcaption}
\DeclareCaptionLabelFormat{gostfigure}{Рисунок #2}
\DeclareCaptionLabelFormat{gosttable}{Табл. #2}
\DeclareCaptionLabelSeparator{gost}{~--~}
\captionsetup{labelsep=gost}
\captionsetup[figure]{labelformat=gostfigure}
\captionsetup[table]{labelformat=gosttable}
\renewcommand{\thesubfigure}{\asbuk{subfigure}}

\usepackage{titlesec}
 
\titleformat{\chapter}[display]
    {\filcenter}
    {\bfseries\texorpdfstring{\MakeUppercase{\chaptertitlename}}{{\chaptertitlename}} \thechapter}
    {4pt}
    {\bfseries\MakeUppercase}{}
 
\titleformat{\section}
    {\normalsize\bfseries}
    {\thesection.}
    {1em}{}
 
\titleformat{\subsection}
    {\normalsize\bfseries}
    {\thesubsection.}
    {1em}{}

% Настройка вертикальных и горизонтальных отступов
\titlespacing*{\chapter}{0pt}{-30pt}{4pt}
\titlespacing*{\section}{\parindent}{*2}{*1}
\titlespacing*{\subsection}{\parindent}{*2}{*1}

\usepackage{geometry}
% \geometry{left=3cm}
% \geometry{right=1.5cm}
% \geometry{top=2.4cm}
% \geometry{bottom=2.4cm}
\geometry{left=2.5cm}
\geometry{right=1cm}
\geometry{top=2cm}
\geometry{bottom=2cm}

\usepackage{enumitem}
\makeatletter
    \def\ukr#1{\expandafter\@ukr\csname c@#1\endcsname}
    \def\@ukr#1{\ifcase#1\or а\or б\or в\or
    г\or ґ\or д\or е\or є\or ж\or з\or и\or і\or ї\or й\or к\or л\or м\or н
    \or о\or п\or р\or с\or т\or у\or ф\or х\or ц\or ч\or ш\or щ\or ь\or ю\or я \fi}
\makeatother
\AddEnumerateCounter{\ukr}{\@ukr}{Українська}
\setlist[enumerate, 1]{label=\ukr*)}
\setlist{nolistsep}
%\renewcommand{\labelitemi}{-}
%\renewcommand{\labelenumi}{\asbuk{enumi})}
%\renewcommand{\labelenumii}{\arabic{enumii})}

\usepackage{tocloft}
\setlength\cftaftertoctitleskip{10pt} % отступ от ЗМІСТ до первого раздела
\renewcommand{\cfttoctitlefont}{\hfill\bfseries\texorpdfstring{\MakeUppercase}{}}
\renewcommand{\cftaftertoctitle}{\hfill}
\renewcommand{\cftbeforetoctitleskip}{0.em}
%\renewcommand{\cftaftertoctitle}{\mbox{}\hfill \\ \mbox{}\hfill{\footnotesize Стр.}\vspace{-2.5em}}
% \renewcommand{\cftchapfont}{\normalsize \texorpdfstring{\MakeUppercase{\chaptername}}{{\chaptername}} }

\renewcommand{\cftchapfont}{}
\renewcommand{\cftchappresnum}{РОЗДІЛ }
\newlength{\xtraspace}
\settowidth{\xtraspace}{\cftchappresnum} % extra space = Chapter + space
\addtolength{\cftchapnumwidth}{\xtraspace} % makes  the indent of the  chapter title larger

\renewcommand\cftchappagefont{\mdseries}
\renewcommand{\cftsecfont}{\hspace{-11pt}}
\renewcommand{\cftsubsecfont}{\hspace{-11pt}}
\renewcommand{\cftbeforechapskip}{0em}
\renewcommand{\cftparskip}{-1mm}
\renewcommand{\cftdotsep}{1}
\renewcommand{\cftchapleader}{\cftdotfill{\cftdotsep}}
\renewcommand{\cftsecaftersnum}{.}
\renewcommand{\cftsubsecaftersnum}{.}
\setcounter{tocdepth}{2} % задать глубину оглавления — до subsection включительно

\newcommand{\likechapterheading}[1]{ 
    \begin{center}
    \textbf{\texorpdfstring{\MakeUppercase{#1}}{{#1}}}
    \end{center}}

\makeatletter
    \newcommand{\l@likechapter}[2]{{\@dottedtocline{0}{0pt}{0pt}{#1}{#2}}}
\makeatother
\newcommand{\likechapter}[1]{    
    \chapter*{#1}
    \addcontentsline{toc}{chapter}{\texorpdfstring{\MakeUppercase{#1}}{{#1}}}
}

\makeatletter
    \newcommand{\l@likesection}[2]{{\@dottedtocline{0}{0pt}{0pt}{#1}{#2}}}
\makeatother
\newcommand{\likesection}[1]{    
    \section*{#1}
    \addcontentsline{toc}{section}{#1}
}

\usepackage[square,numbers,sort&compress]{natbib}
\renewcommand{\bibnumfmt}[1]{#1.\hfill} % нумерация источников в самом списке — через точку
\renewcommand{\bibsection}{\likechapter{Список використаної літератури}} % заголовок специального раздела
\setlength{\bibsep}{0pt}

\usepackage[hidelinks]{hyperref}
\usepackage{microtype}

\usepackage{listings}

\lstdefinestyle{code}{
    basicstyle=\ttfamily\footnotesize,
    breakatwhitespace=false,         
    breaklines=true,                 
    captionpos=b,                    
    keepspaces=true,                            
    numbersep=1pt,                  
    showspaces=false,                
    showstringspaces=false,
    showtabs=false,                  
    tabsize=2
}

\newcommand{\inlinecode}[1]{\texttt{\small #1}}

\usepackage{float}

\newcommand{\append}[1]{
    \phantomsection
    \addcontentsline{toc}{section}{#1}
    \begin{center}
        \textbf{#1}
    \end{center}
}

\usepackage{framed}
\usepackage{dsfont}
\usepackage{xcolor}
\usepgfplotslibrary{fillbetween}
\renewcommand{\emph}[1]{\textit{#1}}
\newtheorem*{corollary*}{Наслідок}

\allowdisplaybreaks[1] % переніс gather
\mathtoolsset{showonlyrefs} % Нумерують тільки ті формули або рівняння, на які є посилання у тексті

% лічильники
\usepackage{lastpage}
\usepackage[figure,table]{totalcount}

\makeatletter
    \AtEndDocument{%
      \immediate\write\@mainaux{%
        \string\gdef\string\totref{\number\value{totreferences}}%
      }%
    }
\makeatother

\usepackage{etoolbox}
% \pretocmd{\chapter}{\addtocounter{totfigures}{\value{figure}}}{}{}
% \pretocmd{\chapter}{\addtocounter{tottables}{\value{table}}}{}{}
% \setcounter{totfigures}{0}

\newcounter{totreferences}
\pretocmd{\bibitem}{\addtocounter{totreferences}{1}}{}{}